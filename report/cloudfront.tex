
%% bare_conf.tex
%% V1.3
%% 2007/01/11
%% by Michael Shell
%% See:
%% http://www.michaelshell.org/
%% for current contact information.
%%
%% This is a skeleton file demonstrating the use of IEEEtran.cls
%% (requires IEEEtran.cls version 1.7 or later) with an IEEE conference paper.
%%
%% Support sites:
%% http://www.michaelshell.org/tex/ieeetran/
%% http://www.ctan.org/tex-archive/macros/latex/contrib/IEEEtran/
%% and
%% http://www.ieee.org/

%%*************************************************************************
%% Legal Notice:
%% This code is offered as-is without any warranty either expressed or
%% implied; without even the implied warranty of MERCHANTABILITY or
%% FITNESS FOR A PARTICULAR PURPOSE! 
%% User assumes all risk.
%% In no event shall IEEE or any contributor to this code be liable for
%% any damages or losses, including, but not limited to, incidental,
%% consequential, or any other damages, resulting from the use or misuse
%% of any information contained here.
%%
%% All comments are the opinions of their respective authors and are not
%% necessarily endorsed by the IEEE.
%%
%% This work is distributed under the LaTeX Project Public License (LPPL)
%% ( http://www.latex-project.org/ ) version 1.3, and may be freely used,
%% distributed and modified. A copy of the LPPL, version 1.3, is included
%% in the base LaTeX documentation of all distributions of LaTeX released
%% 2003/12/01 or later.
%% Retain all contribution notices and credits.
%% ** Modified files should be clearly indicated as such, including  **
%% ** renaming them and changing author support contact information. **
%%
%% File list of work: IEEEtran.cls, IEEEtran_HOWTO.pdf, bare_adv.tex,
%%                    bare_conf.tex, bare_jrnl.tex, bare_jrnl_compsoc.tex
%%*************************************************************************

\documentclass[conference]{IEEEtran}


% \usepackage[caption=false]{caption}
% \usepackage[font=footnotesize]{subfig}
% subfig.sty, also written by Steven Douglas Cochran, is the modern
% replacement for subfigure.sty. However, subfig.sty requires and
% automatically loads Axel Sommerfeldt's caption.sty which will override
% IEEEtran.cls handling of captions and this will result in nonIEEE style
% figure/table captions. To prevent this problem, be sure and preload
% caption.sty with its "caption=false" package option. This is will preserve
% IEEEtran.cls handing of captions. Version 1.3 (2005/06/28) and later 
% (recommended due to many improvements over 1.2) of subfig.sty supports
% the caption=false option directly:
\usepackage[caption=false,font=footnotesize]{subfig}


% *** PDF, URL AND HYPERLINK PACKAGES ***
\usepackage{url}
% url.sty was written by Donald Arseneau. It provides better support for
% handling and breaking URLs. url.sty is already installed on most LaTeX
% systems. The latest version can be obtained at:
% http://www.ctan.org/tex-archive/macros/latex/contrib/misc/
% Read the url.sty source comments for usage information. Basically,
% \url{my_url_here}.

\usepackage{graphicx}
\usepackage{hyperref}
\usepackage{footnote}
\usepackage{listings}
\usepackage{dirtree}

\begin{document}
%
% paper title
% can use linebreaks \\ within to get better formatting as desired
\title{Experimentation with the Amazon CloudFront service}


% author names and affiliations
% use a multiple column layout for up to three different
% affiliations
\author{\IEEEauthorblockN{Timo Saarinen}
\IEEEauthorblockA{School of Science\\
Aalto University\\
Email: timo.l.saarinen@aalto.fi}
\and
\IEEEauthorblockN{Martijn Roo}
\IEEEauthorblockA{School of Science\\
Aalto University\\
Email: martijn.roo@aalto.fi}}


% make the title area
\maketitle

\IEEEpeerreviewmaketitle



\section{Introduction}
Amazon Cloudfront is service that provides a content delivery network (CDN). CDN is a geographically wide network of computer nodes that makes it possible to distribute content to end users with low latency and high data transfer speeds. As of Novemeber 2014, Amazon Cloudfront had edge locations in all continents except Africa and Antarctica. Most of them reside in USA, Europe and Asia. \cite{cloudfront_product_details}

The goal of this experimentation was to examine the latencies with Amazon Cloudfront download distribution feature from different locations in different times. In addition to network latency, we measured also retrieval latency, which is the starting time of a download session. Retrieval latency must not be confused with the total downloading time.

\section{Experimentation setup}
To setup the required cloud services, an account was created to Amazon AWS. Next we created an Amazon S3 bucket for each available location: US East, US West (Oregon), US West (California), Ireland, Frankfurt, Singapore, Sydney, Tokyo, and Sao Paulo. The S3 buckets functioned as storages for the files to be downloaded via CDN. Then, for each of those buckets, we created a Cloudfront download distribution. Thus, all CDN download distributions had their own file origin. All download distributions were set to use only HTTP and to be available at all edge locations around the world. Finally, five different-sized files were uploaded to each S3 bucket. The sizes were 1KB, 10KB, 100KB, 1MB, and 10MB.

\section{Measurement}
asadasd asdasdasdasd dasdasdasd dasdasdasd dasd dasdasdasd asadasd asdasdasdasd dasdasdasd dasdasdasd dasd dasdasdasd asadasd asdasdasdasd dasdasdasd dasdasdasd dasd dasdasdasd asadasd asdasdasdasd dasdasdasd dasdasdasd dasd dasdasdasd asadasd asdasdasdasd dasdasdasd dasdasdasd dasd dasdasdasd asadasd asdasdasdasd dasdasdasd dasdasdasd dasd dasdasdasd asadasd asdasdasdasd dasdasdasd dasdasdasd dasd dasdasdasd asadasd asdasdasdasd dasdasdasd dasdasdasd dasd dasdasdasd asadasd asdasdasdasd dasdasdasd dasdasdasd dasd dasdasdasd asadasd asdasdasdasd dasdasdasd dasdasdasd dasd dasdasdasd asadasd asdasdasdasd dasdasdasd dasdasdasd dasd dasdasdasd asadasd asdasdasdasd dasdasdasd dasdasdasd dasd dasdasdasd asadasd asdasdasdasd dasdasdasd dasdasdasd dasd dasdasdasd asadasd asdasdasdasd dasdasdasd dasdasdasd dasd dasdasdasd asadasd asdasdasdasd dasdasdasd dasdasdasd dasd dasdasdasd asadasd asdasdasdasd dasdasdasd dasdasdasd dasd dasdasdasd

This results in a file structure as follows:\\
\dirtree{%
.1 d2dx94olpiqj0t.cloudfront.net.
.2 california.
.3 1kb.
.4 1kb\_19.11-17.40.txt.
.4 1kb\_19.11-18.40.txt.
.4 ....
.3 10kb.
.3 100kb.
.3 1mb.
.3 10mb.
.3 1kb.
.3 ping\_19.11-17.40.txt.
.3 ping\_19.11-18.40.txt.
.3 ....
.2 home.
.3 ....
.2 ireland.
.3 ....
.2 singapore.
.3 ....
.1 d3kjdkfj34jfd.cloudfront.net.
.2 ....
.1 ....
}

To extract interesting infromation from the data, a Bash script was created. It loops over all the files and builds up a CSV file, exemplified in Table \ref{table:extracteddata}.

\begin{table*}
\renewcommand{\arraystretch}{1.3}
\caption{Extracted data}
\centering
\begin{tabular}{|c|c|c|c|c|c|c|c|c|c|c|}
\hline
\bfseries Measuring location & \bfseries id & \bfseries Datetime & \bfseries S3 location & \bfseries IP address & \bfseries Network latency & \bfseries 1KB (retr. latency) & \bfseries 10kb & \bfseries 100kb & \bfseries 1mb & \bfseries 10mb \\
\hline\hline
home & 0 & 16.11-14.00 & oregon & 54.230.99.246 & 34 & .031158 & .028388 & .028209 & .026788 & .037260 \\
\hline
home & 1 & 16.11-15.00 & oregon & 54.230.98.223 & 41 & .035771 & .035264 & .029972 & .035058 & .027095 \\
\hline
... & ... & ... & ... & ... & ... & ... & ... & ... & ... & ... \\
\hline
\end{tabular}
\label{table:extracteddata}
\end{table*}
\section{Results}
asdasdsaddsa

\section{Analysis}
asdasdsaddsa

% It is trivial that both RTT and download time are lower for uncongested networks than for congested networks in every case.


% An example of a floating figure using the graphicx package.
% Note that \label must occur AFTER (or within) \caption.
% For figures, \caption should occur after the \includegraphics.
% Note that IEEEtran v1.7 and later has special internal code that
% is designed to preserve the operation of \label within \caption
% even when the captionsoff option is in effect. However, because
% of issues like this, it may be the safest practice to put all your
% \label just after \caption rather than within \caption{}.
%
% Reminder: the "draftcls" or "draftclsnofoot", not "draft", class
% option should be used if it is desired that the figures are to be
% displayed while in draft mode.
%
%\begin{figure}[!t]
%\centering
%\includegraphics[width=2.5in]{myfigure}
% where an .eps filename suffix will be assumed under latex, 
% and a .pdf suffix will be assumed for pdflatex; or what has been declared
% via \DeclareGraphicsExtensions.
%\caption{Simulation Results}
%\label{fig_sim}
%\end{figure}

% Note that IEEE typically puts floats only at the top, even when this
% results in a large percentage of a column being occupied by floats.


% An example of a double column floating figure using two subfigures.
% (The subfig.sty package must be loaded for this to work.)
% The subfigure \label commands are set within each subfloat command, the
% \label for the overall figure must come after \caption.
% \hfil must be used as a separator to get equal spacing.
% The subfigure.sty package works much the same way, except \subfigure is
% used instead of \subfloat.
%
%\begin{figure*}[!t]
%\centerline{\subfloat[Case I]\includegraphics[width=2.5in]{subfigcase1}%
%\label{fig_first_case}}
%\hfil
%\subfloat[Case II]{\includegraphics[width=2.5in]{subfigcase2}%
%\label{fig_second_case}}}
%\caption{Simulation results}
%\label{fig_sim}
%\end{figure*}
%
% Note that often IEEE papers with subfigures do not employ subfigure
% captions (using the optional argument to \subfloat), but instead will
% reference/describe all of them (a), (b), etc., within the main caption.


% An example of a floating table. Note that, for IEEE style tables, the 
% \caption command should come BEFORE the table. Table text will default to
% \footnotesize as IEEE normally uses this smaller font for tables.
% The \label must come after \caption as always.
%
%\begin{table}[!t]
%% increase table row spacing, adjust to taste
%\renewcommand{\arraystretch}{1.3}
% if using array.sty, it might be a good idea to tweak the value of
% \extrarowheight as needed to properly center the text within the cells
%\caption{An Example of a Table}
%\label{table_example}
%\centering
%% Some packages, such as MDW tools, offer better commands for making tables
%% than the plain LaTeX2e tabular which is used here.
%\begin{tabular}{|c||c|}
%\hline
%One & Two\\
%\hline
%Three & Four\\
%\hline
%\end{tabular}
%\end{table}


% Note that IEEE does not put floats in the very first column - or typically
% anywhere on the first page for that matter. Also, in-text middle ("here")
% positioning is not used. Most IEEE journals/conferences use top floats
% exclusively. Note that, LaTeX2e, unlike IEEE journals/conferences, places
% footnotes above bottom floats. This can be corrected via the \fnbelowfloat
% command of the stfloats package.



\section{Conclusion}
asdasdadsasd

% conference papers do not normally have an appendix


% trigger a \newpage just before the given reference
% number - used to balance the columns on the last page
% adjust value as needed - may need to be readjusted if
% the document is modified later
%\IEEEtriggeratref{8}
% The "triggered" command can be changed if desired:
%\IEEEtriggercmd{\enlargethispage{-5in}}

% references section

% can use a bibliography generated by BibTeX as a .bbl file
% BibTeX documentation can be easily obtained at:
% http://www.ctan.org/tex-archive/biblio/bibtex/contrib/doc/
% The IEEEtran BibTeX style support page is at:
% http://www.michaelshell.org/tex/ieeetran/bibtex/
%\bibliographystyle{IEEEtran}
% argument is your BibTeX string definitions and bibliography database(s)
%\bibliography{IEEEabrv,../bib/paper}
%
% <OR> manually copy in the resultant .bbl file
% set second argument of \begin to the number of references
% (used to reserve space for the reference number labels box)
% \begin{thebibliography}{1}

% \bibitem{IEEEhowto:kopka}
% H.~Kopka and P.~W. Daly, \emph{Blah blah blah \LaTeX}, 3rd~ed.\hskip 1em plus
%   0.5em minus 0.4em\relax Harlow, England: Addison-Wesley, 1999.

% \end{thebibliography}


\bibliographystyle{IEEEtran}
\bibliography{references}

% that's all folks
\end{document}


