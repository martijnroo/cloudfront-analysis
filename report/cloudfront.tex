
%% bare_conf.tex
%% V1.3
%% 2007/01/11
%% by Michael Shell
%% See:
%% http://www.michaelshell.org/
%% for current contact information.
%%
%% This is a skeleton file demonstrating the use of IEEEtran.cls
%% (requires IEEEtran.cls version 1.7 or later) with an IEEE conference paper.
%%
%% Support sites:
%% http://www.michaelshell.org/tex/ieeetran/
%% http://www.ctan.org/tex-archive/macros/latex/contrib/IEEEtran/
%% and
%% http://www.ieee.org/

%%*************************************************************************
%% Legal Notice:
%% This code is offered as-is without any warranty either expressed or
%% implied; without even the implied warranty of MERCHANTABILITY or
%% FITNESS FOR A PARTICULAR PURPOSE!
%% User assumes all risk.
%% In no event shall IEEE or any contributor to this code be liable for
%% any damages or losses, including, but not limited to, incidental,
%% consequential, or any other damages, resulting from the use or misuse
%% of any information contained here.
%%
%% All comments are the opinions of their respective authors and are not
%% necessarily endorsed by the IEEE.
%%
%% This work is distributed under the LaTeX Project Public License (LPPL)
%% ( http://www.latex-project.org/ ) version 1.3, and may be freely used,
%% distributed and modified. A copy of the LPPL, version 1.3, is included
%% in the base LaTeX documentation of all distributions of LaTeX released
%% 2003/12/01 or later.
%% Retain all contribution notices and credits.
%% ** Modified files should be clearly indicated as such, including  **
%% ** renaming them and changing author support contact information. **
%%
%% File list of work: IEEEtran.cls, IEEEtran_HOWTO.pdf, bare_adv.tex,
%%                    bare_conf.tex, bare_jrnl.tex, bare_jrnl_compsoc.tex
%%*************************************************************************

\documentclass[conference]{IEEEtran}


% \usepackage[caption=false]{caption}
% \usepackage[font=footnotesize]{subfig}
% subfig.sty, also written by Steven Douglas Cochran, is the modern
% replacement for subfigure.sty. However, subfig.sty requires and
% automatically loads Axel Sommerfeldt's caption.sty which will override
% IEEEtran.cls handling of captions and this will result in nonIEEE style
% figure/table captions. To prevent this problem, be sure and preload
% caption.sty with its "caption=false" package option. This is will preserve
% IEEEtran.cls handing of captions. Version 1.3 (2005/06/28) and later
% (recommended due to many improvements over 1.2) of subfig.sty supports
% the caption=false option directly:
\usepackage[caption=false,font=footnotesize]{subfig}


% *** PDF, URL AND HYPERLINK PACKAGES ***
\usepackage{url}
% url.sty was written by Donald Arseneau. It provides better support for
% handling and breaking URLs. url.sty is already installed on most LaTeX
% systems. The latest version can be obtained at:
% http://www.ctan.org/tex-archive/macros/latex/contrib/misc/
% Read the url.sty source comments for usage information. Basically,
% \url{my_url_here}.

\usepackage{graphicx}
\usepackage{hyperref}
\usepackage{footnote}
\usepackage{listings}
\usepackage{dirtree}

\begin{document}
%
% paper title
% can use linebreaks \\ within to get better formatting as desired
\title{File Retrieval Latency in Amazon CloudFront}


% author names and affiliations
% use a multiple column layout for up to three different
% affiliations
\author{\IEEEauthorblockN{Timo Saarinen}
\IEEEauthorblockA{School of Science\\
Aalto University\\
Email: timo.l.saarinen@aalto.fi}
\and
\IEEEauthorblockN{Martijn Roo}
\IEEEauthorblockA{School of Science\\
Aalto University\\
Email: martijn.roo@aalto.fi}}


% make the title area
\maketitle

\IEEEpeerreviewmaketitle


\section{Introduction}
Amazon CloudFront is a content delivery network (CDN) service. A CDN is a geographically wide network of computer nodes that makes it possible to distribute content to end users with low latency and high data transfer speeds. As of November 2014, Amazon CloudFront has a total of 53 edge locations in all continents except Africa and Antarctica. Most of them reside in USA, Europe and Asia \cite{CloudFront_product_details}.

This research aims to achieve a better understanding of how files propagate within Amazon CloudFront after a CloudFront distribution is created.  Multiple files of different sizes are made available through Amazon CloudFront and for these files the network latency and retrieval latency were measured from four different, geographically spread-out locations at multiple times after initiating the CloudFront distribution. Here, the retrieval latency is the time from the first HTTP GET request until the first response packet that contains data. This research is similar to \cite{bermudez2013exploring} but focuses on CloudFront measurements exclusively and not on other Amazon Web Services.


\section{Experiment setup}
To setup the required cloud services, Amazon S3 buckets were created for each of the nine supported locations: US East, US West (Oregon), US West (California), Ireland, Frankfurt, Singapore, Sydney, Tokyo, and Sao Paulo. The S3 buckets function as storages for the files to be distributed and downloaded with CloudFront. For each of those buckets, we created a CloudFront download distribution in that region. Thus, all CDN download distributions had their own S3 origin in the same region. All download distributions were set to use only HTTP and to be available at all supported edge locations. Finally, five files of different sizes were uploaded to the S3 buckets. The files had sizes of 1KB, 10KB, 100KB, 1MB, and 10MB and they were made using rdfc, a Windows utility to make random binary data files of specific sizes.


\section{Measurements}
To capture how files of different sizes propagate through Amazon's CloudFront CDN network, we measured the network latency to the different CloudFront servers as well as the retrieval latency for each of the files at intervals of one hour starting after initializing the CloudFront distributions and continuing for approximately twenty measurements. These measurements were made from a home computer (located in Helsinki, Finland) and from Amazon EC2 instances in California, Ireland and Singapore.

Since we have 9 different CloudFront distributions, 5 different file sizes, about 20 different measurement times, and 4 different measuring locations, there were numerous combinations to take into account. We used a CRON job to execute a bash script every hour at each of the four measurement locations. This bash script pings all the CloudFront distributions and retrieves the different files from all distributions while making a tshark log. The CRON jobs at the different measurements were set up with a small offset (some minutes) from each other in order to not influence each other.

Aggregating all the measurements resulted in the below file structure. The CloudFront distributions are the top-level directories and they contain directories for the four measurement locations. As mentioned before, pings are done to the general CloudFront URL and therefore the ping logs are stored in the measurement location directory. The tshark logs for the five different files are stored each in their own directory for each measurement location. \\

\dirtree{%
.1 d2dx94olpiqj0t.CloudFront.net.
.2 california.
.3 1kb.
.4 1kb\_19.11-17.40.txt.
.4 1kb\_19.11-18.40.txt.
.4 ....
.3 10kb.
.3 100kb.
.3 1mb.
.3 10mb.
.3 1kb.
.3 ping\_19.11-17.40.txt.
.3 ping\_19.11-18.40.txt.
.3 ....
.2 home.
.3 ....
.2 ireland.
.3 ....
.2 singapore.
.3 ....
.1 d3kjdkfj34jfd.CloudFront.net.
.2 ....
.1 ....
}

A second bash script extracts information from the raw logs created during measurements. This resulted in a CSV file containing columns shown in Table \ref{table:extracteddata}. The IP addresses are taken from the first HTTP response packet in the tshark logs of the 10mb file download. The network latencies are averages of pinging ten times to the main CloudFront URL (which is the same for the different files in one distribution). The retrieval latency are the durations between the first HTTP GET request and the first HTTP response packet in the tshark logs.

\begin{table*}
\renewcommand{\arraystretch}{1.3}
\caption{Extracted data}
\centering
\begin{tabular}{|c|c|c|c|c|c|c|c|c|c|c|}
\hline
\bfseries Measuring location & \bfseries id & \bfseries Datetime & \bfseries S3 location & \bfseries IP address & \bfseries Network latency & \bfseries 1KB (retr. latency) & \bfseries 10kb & \bfseries 100kb & \bfseries 1mb & \bfseries 10mb \\
\hline\hline
home & 0 & 16.11-14.00 & oregon & 54.230.99.246 & 34 & .031158 & .028388 & .028209 & .026788 & .037260 \\
\hline
home & 1 & 16.11-15.00 & oregon & 54.230.98.223 & 41 & .035771 & .035264 & .029972 & .035058 & .027095 \\
\hline
... & ... & ... & ... & ... & ... & ... & ... & ... & ... & ... \\
\hline
\end{tabular}
\label{table:extracteddata}
\end{table*}


\section{Results}
In this section the measurement results are presented. Firstly, the network latency is discussed, followed by the retrieval latency.

\subsection{Network latency}

\begin{figure*}[h]
    \centering
    \includegraphics[width=6in]{images/pings_relative.png}
    \caption[]{Relative changes in network latencies}
    \label{fig:ping_relative}
\end{figure*}

Relative changes in network latencies are visualized in \autoref{fig:ping_relative}. For every measurement location, three download distribution from different continents were used: California, Ireland, and Tokyo. The network latencies measured from California decreased 70\% after the first measurement on average. Then they stayed quite constant. Network latency from California EC2 to California download distribution, however, kept being about 30\% more than latency to outside California after the first measurement.

The network latencies from the home laptop (Finland) to Ireland and Tokyo increased by 28\% on average. From home to California, however, latencies increased about 250\%, dropping back momentarily now and then.

From Singapore, the network latencies to California, Ireland and Tokyo kept fluctuating between the initial value and the double of it without a clear trend.

From Ireland, network latencies to all three download distributions kept fluctuating from about 1.5 milliseconds to 85 milliseconds, back and forth. The relative latencies from Ireland to Ireland and Tokyo were so large (around a factor eighty) that they were left out from the figure as not to distort the depiction of the other graphs.

\subsection{Retrieval latency}
Figures \ref{fig:california_relative}, \ref{fig:ireland_relative} and \ref{fig:tokyo_relative} show the retrieval latencies measured for the different file sizes from the four measurement locations to the California, Ireland and Tokyo CloudFront distributions respectively. The figures show the retrieval latency over a twenty-hour timespan. Since we are mostly interested in the relative change of the retrieval latency over time, the latency is depicted as a percentage of the initial measurement per file and measurement location. Due to space limitations, the retrieval latencies for the six other CloudFront distributions are not shown here. They are however similar to the shown graphs and can be retrieved from the authors.

The figures show that the retrieval latency for different files with the same source and destination often show the same trend. The biggest exceptions to this observation are the retrieval latency for the 10 MB file from California to Singapore and for the 1 MB file retrieved from Ireland or Tokyo to Singapore, since those values are continuously significantly higher than that for other files retrieved in Singapore from those destinations.

Another observation is that Ireland's values seem to be either around 100\% or close to 0\% for the California distribution. For the other distributions, Ireland's values have been excluded from the graphs because they fluctuated far more than the other measurements (by a factor of eighty instead of a factor of about four).

A last observation is that the graphs in all three figures do not show a trend for the latency either going down or up and the measurements also do not seem to fluctuate less over time.


\begin{figure*}[]
    \centering
    \includegraphics[width=6in]{images/california_relative.png}
    \caption[]{Relative retrieval times using the California CloudFront distribution}
    \label{fig:california_relative}
\end{figure*}

\begin{figure*}[]
    \centering
    \includegraphics[width=6in]{images/ireland_relative.png}
    \caption[]{Relative retrieval times using the Ireland CloudFront distribution}
    \label{fig:ireland_relative}
\end{figure*}

\begin{figure*}[]
    \centering
    \includegraphics[width=6in]{images/tokyo_relative.png}
    \caption[]{Relative retrieval times using the Tokyo CloudFront distribution}
    \label{fig:tokyo_relative}
\end{figure*}


\section{Analysis}
Below, an analysis of the network latency results is presented, followed by an analysis of the retrieval latency results. Lastly, the relation between network latency and retrieval latency is discussed.

\subsection{Network latency}
The ping requests used to measure the network latencies were always sent to the main CloudFront distribution URL for each distribution. Since the responding host is not always the same for different measurements, we expect that Amazon's DNS servers resolve the main distribution URLs differently sometimes. This could have to do with load balancing. The server that responds to the ping requests to the main distribution URL may not be responsible for delivering any of the files that we may request from that distribution. It may need to know which servers are responsible for delivering those files though and that information may take time to propagate through the CloudFront network.

We expected the server responding to pings to the main distribution URL to stay the same or to change faster than the servers responding to data requests because it only needs to know what servers have the other files without having any of those files. However, Figures \ref{fig:ping_relative} and \ref{fig:california_relative} show for example that there is a correlation between the network and retrieval latency given the same source and destination. This implies that the servers responding to those requests are the same or at least near each other (they have to suffer from the same bottleneck).

Whereas most fluctuations can be explained by changes in network congestion, especially the measurements made by pinging from Ireland to the California distribution seem out of place. The network latency changes in this case between around 1\% to around 100\% and back. It is possible that the network congestion changes, a factor 100 seems a rather large change to be caused by this. Moreover, it is remarkable that the latency seems to assume roughly one of these two values and no values in between.

\subsection{Retrieval latency}
It takes time for files to propagate through the Amazon CloudFront CDN network to all edge-location servers. We would therefore expect the retrieval latency to decrease with time as CDN servers nearer to the requesting node become populated with the requested files. However, as seen in the results section, this does not show in our measurements. Our measurements show continuous fluctuations of the retrieval latency during the twenty-hour measurement interval.

CloudFront could differentiate between files of different sizes by prioritizing the distribution of small files since they occupy less network bandwidth. It could also prioritize larger files because the download speed-up gained with a smaller network latency is much greater with larger files than with smaller files. It would therefore be possible to see structural differences between the retrieval times of files of different sizes. Our measurement data does not show such structural differences however. As stated before, retrieval latencies for different file sizes fluctuate with a similar order of magnitude, given the same measurement location and CloudFront distribution. The exception here is files retrieved from Singapore of either 1 MB (from Ireland or Tokyo) or 10 MB (from California). The retrieval times for these files seem to assume relative latencies significantly higher than the other files retrieved from the same distributions in Singapore. However, from this exception we cannot directly form a general rule.

\subsection{Relation network and retrieval latency}
The retrieval latency is the time between sending the first GET request and receiving the first response packet containing data. As such, the retrieval latency is dependent on the network latency. Besides that, the retrieval latency depends on the time it takes the server answering the request to retrieve and begin serving the file.

Our measurement data often shows a direct correlation between the fluctuations in network latency and the fluctuations in retrieval latency. The retrieval time from Ireland to the California distribution shows a very similar pattern to the network latency in that same path. The same holds for latencies measured from California and Singapore to the Ireland, Tokyo and California distributions and for those measured from Home to Ireland and Tokyo. The most notable exception to this general rule is between Home and the California distribution since fluctuations of up to a factor four seem to have no noticeable impact on the retrieval time. The network latency is shown in proportion to the initial measurement, which means that a factor four increase can be rather small in absolute terms and this is the case for the Home-California connection.


\section{Conclusion}
This research set out to examine file propagation in Amazon CloudFront, however, our data appears unsuitable to draw clear conclusions about file propagation. We expected to see fluctuations in network latency and retrieval times due to network congestion and we expected a trend of both latencies decreasing with time as the files would propagate through the CloudFront distributions to more and more edge locations.

Whereas we measured fluctuations that are likely caused by changes in network congestion as expected, we saw no trend of decreasing latencies. It could be that CloudFront propagated our files to edge locations much faster than expected, causing us to measure only the edge nodes near the measurement locations. It is also possible that propagation takes much longer than expected causing us to only measure CloudFront servers near the original S3 source.

Another unexpected outcome were some patterns in network latency measurements made from an Amazon EC2 instance in Ireland. These patterns seemed to fluctuate heavily (around a factor 100) also when requesting from the Ireland instance to the Ireland distribution. This is hard to explain with network congestion alone and might in fact be an anomaly with that particular EC2 instance.

Future work could perform measurements with smaller intervals to gain more detailed measurements. It could also measure the total download time next to the network and retrieval latency. This is related to the network and retrieval latency, so changes between the change in total download time and the network and retrieval latency could give new insights. Lastly, future work could compare latencies for conventional downloading not using CloudFront to latencies measured with CloudFront.

% conference papers do not normally have an appendix

\bibliographystyle{IEEEtran}
\bibliography{references}

% that's all folks
\end{document}
